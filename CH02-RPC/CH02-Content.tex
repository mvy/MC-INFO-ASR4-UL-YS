\title{Systèmes d'exploitation, 2ème année}
\subtitle{Appel de procédures distantes}

\author{Yves \textsc{Stadler}}
\institute{Université de Lorraine - IUT de Metz}

\date{\today}

\begin{document}

%%
% Page de Titre
%%
\begin{frame}
\titlepage
\end{frame}


\def\sectitle{Agenda}
\section{\sectitle}
% Frame
\begin{frame}{\sectitle}
    % Block
    \def\subsectitle{Plan du cours}
    \subsection{\subsectitle}
    \begin{block}{\subsectitle}
        \begin{itemize}
            \item Qu'est-ce que les RPC
            \item Usage
            \item Implémenation en C
        \end{itemize}
    \end{block}
\end{frame}


\def\sectitle{Principe des RPC}
\section{\sectitle}
% Frame
\begin{frame}[containsverbatim]{\sectitle}
    % Block
    \def\subsectitle{Transparence}
    \subsection{\subsectitle}
    \begin{block}{\subsectitle}
        \begin{itemize}
            \item L'objectif des appels de procédures distantes:
                \begin{itemize}
                    \item Fournir un appel de fonction (\verb+func(a,b)+)
                    \item S'exécuter à distance
                    \item Renvoyer un résultat au programme appelant.
                \end{itemize}
        \end{itemize}
    \end{block}
    % Block
    \def\subsectitle{Avantage}
    \subsection{\subsectitle}
    \begin{block}{\subsectitle}
        \begin{itemize}
            \item Le programmeur ne se soucie pas de la gestion du réseau!
                \verb+\o/+
            \item On peut proposer une liste de service réseaux
            \item Le protocol est assisté
        \end{itemize}
    \end{block}
\end{frame}


\def\sectitle{Définir le protocol}
\section{\sectitle}
% Frame
\begin{frame}[containsverbatim]{\sectitle}
    % Block
    \def\subsectitle{Langage RPC}
    \subsection{\subsectitle}
    \begin{exampleblock}{\subsectitle}
        \begin{verbatim}
  1 enum result_t {SUCESS, FAILURE};
  2 
  3 struct etudiant {
  4     string name<>;
  5     int age;
  6 };
  7 
  8 program ETUDDB {
  9     version ETUDDB_V1 {
 10         result_t ADD_ETUD(etudiant) = 1;
 11         result_t PRINT() = 2;
 12     } = 1;
 13 } = 0x2fffffff;
        \end{verbatim}
    \end{exampleblock}
\end{frame}
\end{document}

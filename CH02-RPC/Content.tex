\title{Systèmes d'exploitation, 2ème année}
\subtitle{Appel de procédures distantes}

\author{Yves \textsc{Stadler}}
\institute{Université de Lorraine - IUT de Metz}

\date{\today}

\begin{document}

%%
% Page de Titre
%%
\begin{frame}
\titlepage
\end{frame}

\def\sectitle{Agenda}
\section{\sectitle}
% Frame
\begin{frame}{\sectitle}
    % Block
    \def\subsectitle{Plan du cours}
    \subsection{\subsectitle}
    \begin{block}{\subsectitle}
        \begin{itemize}
            \item Principe des RPC
            \item Outils connus
            \item Stratégies d'exécution
            \item Problèmes
            \item Sérialisation
            \item Gestion des données
            \item Historique
            \item Implémentation
        \end{itemize}
    \end{block}
\end{frame}

\def\sectitle{Principe des RPC}
\section{\sectitle}
% Frame
\begin{frame}[containsverbatim]{\sectitle}
    % Block
    \def\subsectitle{Transparence}
    \subsection{\subsectitle}
    \begin{block}{\subsectitle}
        \begin{itemize}
            \item L'objectif des appels de procédures distantes:
                \begin{itemize}
                    \item Fournir un appel de fonction (\verb+func(a,b)+)
                    \item S'exécuter à distance
                    \item Renvoyer un résultat au programme appelant.
                \end{itemize}
        \end{itemize}
    \end{block}
    % Block
    \def\subsectitle{Avantage}
    \subsection{\subsectitle}
    \begin{block}{\subsectitle}
        \begin{itemize}
            \item Le programmeur ne se soucie pas de la gestion du réseau!
                \verb+\o/+
            \item On peut proposer une liste de service réseaux
            \item Le protocole est assisté
        \end{itemize}
    \end{block}
\end{frame}

\begin{frame}{\sectitle}
    \includegraphics[width=\textwidth]{images/RPC_overview.png}
\end{frame}

\def\sectitle{Outils connus}
\section{\sectitle}
% Frame
\begin{frame}{\sectitle}
    % Block
    \def\subsectitle{Traditionnels}
    \subsection{\subsectitle}
    \begin{block}{\subsectitle}
        \begin{itemize}
            \item Sun ONC/RPC
            \item OSF DCE (Distributed Computing Environment)
            \item Procédures stockée des BdD
        \end{itemize}
    \end{block}
    % Block
    \def\subsectitle{Objets répartis}
    \subsection{\subsectitle}
    \begin{block}{\subsectitle}
        \begin{itemize}
            \item CORBA (Common Object Request Broker Architecture)
            \item Java RMI (Remote Method Invocation)
            \item DCOM (Distributed Common Object Model)
        \end{itemize}
    \end{block}
\end{frame}

% Frame
\begin{frame}{\sectitle}
    % Block
    \def\subsectitle{Systèmes de composants}
    \subsection{\subsectitle}
    \begin{block}{\subsectitle}
        \begin{itemize}
            \item Sun  J2EE EJB (Entreprise Java Beans)
            \item Corba Component Model
            \item WS-SOAP (WebServices Simple Object Access Protocol)
            \item Implémentations souvent incompatibles
        \end{itemize}
    \end{block}
\end{frame}

\begin{frame}{\sectitle}
    % Block
    \def\subsectitle{Séquence}
    \subsection{\subsectitle}
    \begin{block}{\subsectitle}
        \begin{itemize}
            \item Le client appelle la souche (stub) appel local
            \item La souche formatte les paramètres dans un message
            \item Le système envoi le message
            \item Le système serveur reçoit le message
            \item La souche serveur récupère le message
            \item Le serveur appelle la procédure choisie.
        \end{itemize}
    \end{block}
\end{frame}

\def\sectitle{Stratégie d'exécution}
\section{\sectitle}
% Frame
\begin{frame}{\sectitle}
    % Block
    \def\subsectitle{Stratégies}
    \subsection{\subsectitle}
    \begin{block}{\subsectitle}
        \begin{itemize}
            \item Le code est déporté sur un serveur
            \item Qui l'exécute?
            \item Comment sont gérés les données?
        \end{itemize}
    \end{block}
\end{frame}
% Frame
\begin{frame}{\sectitle}
    % Block
    \def\subsectitle{Migration / Rapatriement}
    \subsection{\subsectitle}
    \begin{block}{\subsectitle}
        \begin{itemize}
            \item On amène le code sur la machine locale
        \end{itemize}
    \end{block}
    % Block
    \def\subsectitle{Critiques}
    \subsection{\subsectitle}
    \begin{block}{\subsectitle}
        \begin{itemize}
            \item Efficaces
            \item Problème avec systèmes hétérogènes
            \item La taille du code fait varier les performances
        \end{itemize}
    \end{block}
\end{frame}

% Frame
\begin{frame}{\sectitle}
    % Block
    \def\subsectitle{Mémoire virtuelle}
    \subsection{\subsectitle}
    \begin{block}{\subsectitle}
        \begin{itemize}
            \item L'appel côté client génère un défaut de page
            \item On va chercher la page sur le serveur
        \end{itemize}
    \end{block}
    % Block
    \def\subsectitle{Critiques}
    \subsection{\subsectitle}
    \begin{block}{\subsectitle}
        \begin{itemize}
            \item Efficaces pour de nombreux appels
            \item On peut utiliser des pointeurs
            \item Systèmes homogènes requis
            \item La mémoire répartie doit rester cohérente.
        \end{itemize}
    \end{block}
\end{frame}

% Frame
\begin{frame}{\sectitle}
    % Block
    \def\subsectitle{Messages asynchrones}
    \subsection{\subsectitle}
    \begin{block}{\subsectitle}
        \begin{itemize}
            \item Un gestionnaire client intercepte la demande (wrapper)
            \item Le wrapper gère l'appel distant et renvoie le résultat
            \item Le serveur reçoit les infos de l'appel et l'exécute
            \item Le serveur renvoie la réponse au wrapper.
        \end{itemize}
    \end{block}
    % Block
    \def\subsectitle{Critiques}
    \subsection{\subsectitle}
    \begin{block}{\subsectitle}
        \begin{itemize}
            \item La taille ne compte pas
            \item Hétérogénéité possible
            \item Transactionel 
            \item Pas de pointeurs
            \item Échanges de types complexes compliqué
            \item Peu efficace pour beaucoup d'appels
        \end{itemize}
    \end{block}
\end{frame}

% Frame
\begin{frame}{\sectitle}
    % Block
    \def\subsectitle{RPC légers}
    \subsection{\subsectitle}
    \begin{block}{\subsectitle}
        \begin{itemize}
            \item On se passe des wrappers
        \end{itemize}
    \end{block}
    % Block
    \def\subsectitle{Critiques}
    \subsection{\subsectitle}
    \begin{block}{\subsectitle}
        \begin{itemize}
            \item Uniquement en local
            \item Si threads, ok 
            \item Si fork, Segment de mémoire partagée
        \end{itemize}
    \end{block}
\end{frame}

\def\sectitle{Problème des RPC}
\section{\sectitle}
% Frame
\begin{frame}{\sectitle}
    % Block
    \def\subsectitle{Client bloqué - Mode asynchrone}
    \subsection{\subsectitle}
    \begin{block}{\subsectitle}
        \begin{itemize}
            \item Mon client est bloqué lors des appels
            \item Solution: créer un thread pour l'exécuter pendant que le
                client continu son boulot.
            \item Le client récupère l'information quand ça l'arrange.
        \end{itemize}
    \end{block}
\end{frame}

% Frame
\begin{frame}{\sectitle}
    % Block
    \def\subsectitle{Serveur bloqué - Parallélisation du serveur}
    \subsection{\subsectitle}
    \begin{block}{\subsectitle}
        \begin{itemize}
            \item Le serveur peut-être en mode séquentiel
                \begin{itemize}
                    \item les requêtes s'exécutent une à la suite de l'autre
                    \item si il y a blocage, tout le monde est ralentit
                \end{itemize}
            \item Si le serveur est en mode parallèle, il faut gérer la
                concurrence!
                \begin{itemize}
                    \item Utilisation de sémaphores!
                \end{itemize}
        \end{itemize}
    \end{block}
\end{frame}

% Frame
\begin{frame}{\sectitle}
    % Block
    \def\subsectitle{Passage de paramètres}
    \subsection{\subsectitle}
    \begin{block}{\subsectitle}
        \begin{itemize}
            \item Passage par référence impossible
            \item Conversion des données (standardisation)
        \end{itemize}
    \end{block}
    % Block
    \def\subsectitle{Traitement des erreurs}
    \subsection{\subsectitle}
    \begin{block}{\subsectitle}
        \begin{itemize}
            \item Temps de réponse trop long 
                \begin{itemize}
                    \item Appel perdu
                    \item Réponse perdue
                    \item Défaillance serveur
                \end{itemize}
            \item Résolution
                \begin{itemize}
                    \item Le client renvoie la demande
                    \item Problèmes (double réponses, etc.)
                \end{itemize}
        \end{itemize}
    \end{block}
\end{frame}

% Frame
\begin{frame}{\sectitle}
    % Block
    \def\subsectitle{Sans état}
    \subsection{\subsectitle}
    \begin{block}{\subsectitle}
        \begin{itemize}
            \item Le serveur ne peut pas gérer d'états
            \item Il faut enregistrer les données qui doivent être persistantes
            \item Toute requête doit contenir tout ce qu'il faut pour exécuter
                correctement les appels
            \item Prévoir des fonctions qui ont toujours le même effet avec les
                même paramètres.
        \end{itemize}
    \end{block}
\end{frame}

\def\sectitle{Serialisation}
\section{\sectitle}
% Frame
\begin{frame}{\sectitle}
    % Block
    \def\subsectitle{Usage}
    \subsection{\subsectitle}
    \begin{block}{\subsectitle}
        \begin{itemize}
            \item Stockage d'objet sous une forme exploitable
            \item Fichier, RPC, réseau, inter-langage. 
            \item Conventions
            \item Marshalling, Serialization, deflating
        \end{itemize}
    \end{block}
    % Block
    \def\subsectitle{Fonctions}
    \subsection{\subsectitle}
    \begin{block}{\subsectitle}
        \begin{itemize}
            \item Fournies par beaucoup de framework/langages 
            \item Manuelle (écriture de flux binaires)
        \end{itemize}
    \end{block}
\end{frame}

\def\sectitle{Gestion des données}
\section{\sectitle}
% Frame
\begin{frame}{\sectitle}
    % Block
    \def\subsectitle{Stateless}
    \subsection{\subsectitle}
    \begin{block}{\subsectitle}
        \begin{itemize}
            \item Les paramètres sont suffisants pour générer la sortie (maths)
            \item Cas simple, peu de problème
        \end{itemize}
    \end{block}
    % Block
    \def\subsectitle{Manipulation de données}
    \subsection{\subsectitle}
    \begin{block}{\subsectitle}
        \begin{itemize}
            \item Contrôle de concurrence
            \item Que se passe-t-il si panne?
            \item Il faut gérer des transactions
        \end{itemize}
    \end{block}
\end{frame}

\def\sectitle{Information d'historique}
\section{\sectitle}
% Frame
\begin{frame}{\sectitle}
    % Block
    \def\subsectitle{Sans état}
    \subsection{\subsectitle}
    \begin{block}{\subsectitle}
        \begin{itemize}
            \item Une opération n'a pas besoin de connaître des informations
                d'une opération précédente
        \end{itemize}
    \end{block}
    % Block
    \def\subsectitle{Avec état}
    \subsection{\subsectitle}
    \begin{block}{\subsectitle}
        \begin{itemize}
            \item Nécessite des informations historiques
            \item Usage d'un descriptif pour chaque relation client-serveur
        \end{itemize}
    \end{block}
\end{frame}

\def\sectitle{Implémentation}
\section{\sectitle}
% Frame
\begin{frame}[containsverbatim]{\sectitle}
    % Block
    \def\subsectitle{Langage RPC - définition du protocole}
    \subsection{\subsectitle}
    \begin{exampleblock}{\subsectitle}
        \begin{verbatim}
  1 enum result_t {SUCESS, FAILURE};
  2 
  3 struct etudiant {
  4     string name<>;
  5     int age;
  6 };
  7 
  8 program ETUDDB {
  9     version ETUDDB_V1 {
 10         result_t ADD_ETUD(etudiant) = 1;
 11         result_t PRINT() = 2;
 12     } = 1;
 13 } = 0x2fffffff;
        \end{verbatim}
    \end{exampleblock}
\end{frame}

% Frame
\begin{frame}[containsverbatim]{\sectitle}
    % Block
    \def\subsectitle{External Data Representation (XDR)}
    \subsection{\subsectitle}
    \begin{block}{\subsectitle}
        \begin{itemize}
            \item Un fichier \verb+nom_xdr.c+ contenant les types définis
            \item Ne pas modifier, inclure le \verb+.o+
            \item RFC 1831
        \end{itemize}
    \end{block}
    % Block
    \def\subsectitle{Server Stub}
    \subsection{\subsectitle}
    \begin{block}{\subsectitle}
        \begin{itemize}
            \item Contient la gestion des appels client et la liaison avec le
                service
            \item Appel des fonctions non implémentées
            \item Ne pas modifier, inclure le \verb+.o+
        \end{itemize}
    \end{block}
\end{frame}

\def\sectitle{}
\subsection{\sectitle}
%%Frame
\begin{frame}[containsverbatim]{\sectitle}
    % Block
    \def\subsectitle{Client Stub}
    \subsection{\subsectitle}
    \begin{block}{\subsectitle}
        \begin{itemize}
            \item Contient les appels aux fonctions définies précédemment
            \item Elle appelles les fonction serveur
            \item Ne pas modifier, inclure le \verb+.o+
        \end{itemize}
    \end{block}
    % Block
    \def\subsectitle{Header}
    \subsection{\subsectitle}
    \begin{block}{\subsectitle}
        \begin{itemize}
            \item Définis les fonctions
            \item inclure le \verb+.h+
        \end{itemize}
    \end{block}
\end{frame}

\end{document}

%%%%\%%%%%%%%%%%%%%%%%%%%%%%%%%%%%%%%%%%%%%%%%%%
%% Introduction aux Systèmes d'exploitation  %%
%%   * Historique                            %%
%%   * Principes fondamentaux                %%
%%   * Grandes classes de systèmes           %%
%%%%%%%%%%%%%%%%%%%%%%%%%%%%%%%%%%%%%%%%%%%%%%%

\title{Systèmes d'exploitation, 2ème année}
\subtitle{Multi-Threading}

\author{Yves \textsc{Stadler}}
\institute{Université de Lorraine - IUT de Metz}

\date{\today}

\begin{document}

%%
% Page de Titre
%%
\begin{frame}
\titlepage
\end{frame}

\def\sectitle{Agenda}
\section{\sectitle}
\def\subsectitle{Plan du cours}
\subsection{\subsectitle}

\begin{frame}{\sectitle}
\begin{block}{\subsectitle}
\begin{itemize}
    \item Différence entre Threads et Processus
    \item Utilisation des threads
    \item Synchronisation et ordonnancement
    \item Implémentation du multi-threading avec pthread
\end{itemize}
\end{block}
\end{frame}


\def\sectitle{Différence entre Threads et Processus}
\section{\sectitle}
\begin{frame}{\sectitle}
    \def\subsectitle{Points communs}
    \subsection{\subsectitle}
    \begin{block}{\subsectitle}
        \begin{itemize}
            \item Permet d'obtenir plusieurs instructions s'exécutant en
                parallèle.
            \item 
        \end{itemize}
    \end{block}

    \def\subsectitle{Différences}
    \subsection{\subsectitle}
    \begin{block}
        \begin{itemize}
            \item Les threads partages leurs mémoire
            \item Les threads ne sont pas ordonnancés par le système
        \end{itemize}
    \end{block}
\end{frame}



\end{document}

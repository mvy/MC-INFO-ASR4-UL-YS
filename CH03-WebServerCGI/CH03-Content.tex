\title{Systèmes d'exploitation, 2ème année}
\subtitle{Architecture d'un serveur web - CGI}

\author{Yves \textsc{Stadler}}
\institute{Université de Lorraine - IUT de Metz}

\date{\today}

\begin{document}

%%
% Page de Titre
%%
\begin{frame}
\titlepage
\end{frame}

\def\sectitle{Agenda}
\section{\sectitle}
\def\subsectitle{Plan du cours}
\subsection{\subsectitle}

\begin{frame}{\sectitle}
\begin{block}{\subsectitle}
\begin{itemize}
\item Système de gestion des fichiers
\end{itemize}
\end{block}
\end{frame}

\def\sectitle{Client - Serveur}
\section{\sectitle}
% Frame
\begin{frame}{\sectitle}
    % Block
    \def\subsectitle{Modèle - Rappels}
    \subsection{\subsectitle}
    \begin{block}{\subsectitle}
        \begin{itemize}
            \item Mode de communication
            \item Le client envoi des requêtes
            \item Le serveur répond aux requêtes
            \item Communiquent conformément au modèle OSI.
        \end{itemize}
    \end{block}
\end{frame}

% frame
\begin{frame}{\sectitle}
    % block
    \def\subsectitle{Serveurs}
    \subsection{\subsectitle}
    \begin{block}{\subsectitle}
        \begin{itemize}
            \item mainframe
            \item peer to peer
            \item architecture n-tiers
        \end{itemize}
    \end{block}
    % Block
    \def\subsectitle{Clients}
    \subsection{\subsectitle}
    \begin{block}{\subsectitle}
        \begin{itemize}
            \item Légers (ne fait que des appels et affiche)
            \item Lourd (réception des données, travail conjoint)
            \item Riches (Utilisation de réponse partielles)
        \end{itemize}
    \end{block}
\end{frame}

\def\sectitle{Serveur Web}
\section{\sectitle}
% frame
\begin{frame}{\sectitle}
    % block
    \def\subsectitle{Objectif}
    \subsection{\subsectitle}
    \begin{block}{\subsectitle}
        \begin{itemize}
            \item fournir des informations ou un service
        \end{itemize}
    \end{block}
    % block
    \def\subsectitle{Acteurs}
    \subsection{\subsectitle}
    \begin{block}{\subsectitle}
        \begin{itemize}
            \item utilisateurs
            \item gestionnaire de connexion
            \item gestion des données (disque, SGBD)
            \item exécutant (api, programme, module)
        \end{itemize}
    \end{block}
\end{frame}

\def\sectitle{Fonctionnement}
\section{\sectitle}
% Frame
\begin{frame}{\sectitle}
    % Block
    \def\subsectitle{Usage}
    \subsection{\subsectitle}
    \begin{block}{\subsectitle}
        \begin{itemize}
            \item Les API permettent d'écrire des modules qui s'exécutent sur le
                serveur
            \item Le serveur intercepte les requêtes HTTP et les transmet à
                l'application.
        \end{itemize}
    \end{block}
\end{frame}

% Frame
\begin{frame}{\sectitle}
    % Block
    \def\subsectitle{Fonctionnement}
    \subsection{\subsectitle}
    \begin{block}{\subsectitle}
        \begin{itemize}
            \item En général une requête correspond à une instance du serveur.
            \item Favorise la réutilisation des applications clients/serveurs
                existantes
            \item Cela nécessite quelques ajouts aux fonctions de bases du
                serveur:
                \begin{itemize}
                    \item gestion des sessions
                    \item gestion des droits
                    \item gestion de la présentation
                \end{itemize}
        \end{itemize}
    \end{block}
\end{frame}


% Frame
\begin{frame}{\sectitle}
    % Block
    \def\subsectitle{Modules}
    \subsection{\subsectitle}
    \begin{block}{\subsectitle}
        \begin{itemize}
            \item Modules d'interprétation
            \item Standardisation du développement web
            \item Facilité de réutilisation des fonctions communes.
        \end{itemize}
    \end{block}
\end{frame}


\def\sectitle{Common GateWay Interface}
\section{\sectitle}
% Frame
\begin{frame}{\sectitle}
    % Block
    \def\subsectitle{Principe}
    \subsection{\subsectitle}
    \begin{block}{\subsectitle}
        \begin{itemize}
            \item Description d'une passerelle entre le serveur et des
                applications externes
            \item Le serveur lance un programme au lieu d'aller chercher une
                page
            \item La sortie standard d'un script CGI est redirigée vers le
                client.
        \end{itemize}
    \end{block}
    % Block
    \def\subsectitle{En sortie}
    \subsection{\subsectitle}
    \begin{block}{\subsectitle}
        \begin{itemize}
            \item Une réponse HTTP valide
            \item Redirection, fichier, page HTML.
        \end{itemize}
    \end{block}
\end{frame}

% Frame
\begin{frame}{\sectitle}
    % Block
    \def\subsectitle{Entrées sorties}
    \subsection{\subsectitle}
    \begin{block}{\subsectitle}
        \begin{itemize}
            \item Les entrées se trouvent dans la Query String. Celle-ci se
                trouve dans les variables d'environnement que le serveur va
                fixer au moment du lancement.
            \item Si la requête est utilise la méthode POST, il n'y a pas de
                Query String, il faut utiliser stdin.
            \item La variable Content\_length contient la taille du message
            \item Il est préférable d'utiliser le POST car pas de problème de
                cache
        \end{itemize}
    \end{block}
\end{frame}

% Frame
\begin{frame}{\sectitle}
    % Block
    \def\subsectitle{Attention}
    \subsection{\subsectitle}
    \begin{block}{\subsectitle}
        \begin{itemize}
            \item Un script/programme CGI va être exécuté sur votre serveur
            \item Attention à ce que l'on programme.
            \item Valider les données d'entrées (tailles, contenu)
            \item Vérifier que la requête provient bien de notre serveur
            \item Éviter les appels aux commandes systèmes.
        \end{itemize}
    \end{block}
\end{frame}

\def\sectitle{Apache - Configuration}
\section{\sectitle}
% Frame
\begin{frame}[containsverbatim]{\sectitle}
    % Block
    \def\subsectitle{}
    \subsection{\subsectitle}
    \begin{block}{\subsectitle}
        \begin{verbatim}
-rw-r--r-- 1 root root  7994 Jan  1  2011 apache2.conf
drwxr-xr-x 2 root root  1024 Feb 13 01:01 conf.d
-rw-r--r-- 1 root root  1169 Jan  1  2011 envvars
-rw-r--r-- 1 root root     0 Aug 17  2010 httpd.conf
-rw-r--r-- 1 root root 31063 Jan  1  2011 magic
drwxr-xr-x 2 root root  9216 Feb 13 01:01 mods-available
drwxr-xr-x 2 root root  1024 Apr 18  2011 mods-enabled
-rw-r--r-- 1 root root   762 Feb 10 16:45 ports.conf
drwxr-xr-x 2 root root  1024 Feb 13 01:01 sites-available
drwxr-xr-x 2 root root  1024 Feb 10 17:00 sites-enabled
        \end{verbatim}
    \end{block}
\end{frame}

% Frame
\begin{frame}[containsverbatim]{\sectitle}
    % Block
    \def\subsectitle{/etc/apache2}
    \subsection{\subsectitle}
    \begin{block}{\subsectitle}
        \begin{itemize}
            \item \verb+xxx-available+ configuration des mods/sites
            \item \verb+xxx-enabled+ liens vers le précédent
        \end{itemize}
    \end{block}
    % Block
    \def\subsectitle{Paramètrages du sith}
    \subsection{\subsectitle}
    \begin{block}{\subsectitle}
        \begin{itemize}
            \item Défini la configuration d'un site (admin, mail, root)
            \item Lie les URL aux dossiers locaux
            \item Indique les droits pour ce site (Options utilisables, override
                possible ou non, IP autorisées)
            \item Décide de l'activation des CGI et leur dossier.
            \item Défini le niveau de debug.
        \end{itemize}
    \end{block}
\end{frame}

% Frame
\begin{frame}{\sectitle}
    % Block
    \def\subsectitle{VirtualHost}
    \subsection{\subsectitle}
    \begin{block}{\subsectitle}
        \begin{itemize}
            \item Un virtual host est un serveur web hébergé
            \item Il peut y en avoir plusieurs sur le même serveur (c'est
                l'intérêt)
            \item Il doit définir le nom du serveur (URL)
            \item Le nom de l'administrateur (mail)
            \item La racine où se trouver les fichiers
            \item Éventuellement les fichiers de logs.
        \end{itemize}
    \end{block}
\end{frame}

% Frame
\begin{frame}[containsverbatim]{\sectitle}
    % Block
    \def\subsectitle{Directory}
    \subsection{\subsectitle}
    \begin{block}{\subsectitle}
        \begin{itemize}
            \item La balise directory permet d'appliquer des directives à un
                répertoire particuliers
            \item Activer les CGI
            \item Activer le suivi des liens symboliques
            \item Demande un password
            \item Autoriser l'utilisation des .htaccess
        \end{itemize}
    \end{block}
    % Block
    \def\subsectitle{Exemple}
    \subsection{\subsectitle}
    \begin{exampleblock}{\subsectitle}
        \begin{verbatim}
<Directory "/home/mvy/git/serveur">
        Options Indexes FollowSymLinks MultiViews +ExecCGI
        AllowOverride None
        Order allow,deny
        allow from all
</Directory>
        \end{verbatim}
    \end{exampleblock}
\end{frame}

\def\sectitle{CGI}
\section{\sectitle}
% Frame
\begin{frame}{\sectitle}
    % Block
    \def\subsectitle{Alias}
    \subsection{\subsectitle}
    \begin{block}{\subsectitle}
        \begin{itemize}
            \item Créer un alias entre une adresse URL et un dossier
            \item Permet d'avoir des fichiers ailleurs que dans la DocumentRoot
            \item Il faut aussi donner les droits ensuite
        \end{itemize}
    \end{block}
    % Block
    \def\subsectitle{ScriptAlias}
    \subsection{\subsectitle}
    \begin{block}{\subsectitle}
        \begin{itemize}
            \item Comme pour alias, mais permet de stocker des CGI
            \item Pas besoin d'utiliser les handlers.
            \item Appel du module mod\_cgi
        \end{itemize}
    \end{block}
\end{frame}

% Frame
\begin{frame}[containsverbatim]{\sectitle}
    % Block
    \def\subsectitle{Divers}
    \subsection{\subsectitle}
    \begin{block}{\subsectitle}
        \begin{itemize}
            \item Les handlers permettent de faire reconnaitre une extension en
                tant que script à exécuter
            \item Les scripts interprétés doivent contenir le nom de
                l'interpréteur (\verb+#!/bin/bash+)
        \end{itemize}
    \end{block}
\end{frame}

% Frame
\begin{frame}{\sectitle}
    % Block
    \def\subsectitle{Fast-CGI}
    \subsection{\subsectitle}
    \begin{block}{\subsectitle}
        \begin{itemize}
            \item Le fast-cgi de mod\_fastcgi permet de faire du CGI avec
                persistance
            \item Attention à la concurrence.
        \end{itemize}
    \end{block}
\end{frame}

\def\sectitle{CGI en programmation}
\section{\sectitle}
% Frame
\begin{frame}[containsverbatim]{\sectitle}
    % Block
    \def\subsectitle{Sortie}
    \subsection{\subsectitle}
    \begin{block}{\subsectitle}
        \begin{itemize}
            \item Il faut que le CGI comment par donner le type de réponse 
            \item \verb+Content-type: text/html+ par exemple
            \item Ensuite on peut écrire du HTML
            \item Beaucoup de types sont possibles, du fichiers à l'audio.
        \end{itemize}
    \end{block}
    % Block
    \def\subsectitle{Entrée}
    \subsection{\subsectitle}
    \begin{block}{\subsectitle}
        \begin{itemize}
            \item Les variables : récupérables avec \verb+getenv+
            \item Attention URL encodé
        \end{itemize}
    \end{block}
\end{frame}

% Frame
\begin{frame}{\sectitle}
    % Block
    \def\subsectitle{Bibliothèque}
    \subsection{\subsectitle}
    \begin{block}{\subsectitle}
        \begin{itemize}
            \itemEn C : cgic
            \item En Perl: CGI.pm, CGI::Lite
        \end{itemize}
    \end{block}
\end{frame}

\end{document}
